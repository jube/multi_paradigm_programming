\part{Introduction}

\section{Introduction}

\subsection{About this course}

\begin{frame}{Objective}{}
  \begin{block}{Objective}
    Learn how to use different paradigms in a single application with modern \CCLang

    \begin{itemize}
    \item
      procedural programming
    \item
      object-oriented programming
    \item
      functional programming
    \item
      generic programming
    \end{itemize}
  \end{block}
\end{frame}

\begin{frame}{Organisation}{}
  \begin{block}{Hours}
    \begin{itemize}
    \item
      Lecture: 5 $\times$ 1h30
    \item
      Test: 1 $\times$ 1h30
    \item
      Laboratory: 6 $\times$ 3h00
    \end{itemize}
  \end{block}

  \begin{block}{Evaluation}
    \begin{itemize}
    \item
      Multiple choice test (50\%)
    \item
      Laboratory project in \CCLang (50\%)
    \end{itemize}
  \end{block}
\end{frame}

\begin{frame}{Resources}{}
  \begin{block}{Online}
    \begin{itemize}
    \item
      \CCLang Reference: \url{http://en.cppreference.com/w/cpp}
    \item
      \CCLang FAQ: \url{http://isocpp.org/faq}
    \end{itemize}
  \end{block}

  \begin{block}{Books}
    \begin{thebibliography}{}
    \bibitem{Stroustrup}
      Bjarne Stroustrup.
      \newblock {\em The \CCLang Programming Language}.
      \newblock 4th edition, 2013, Addison--Wesley
    \end{thebibliography}
  \end{block}
\end{frame}

\section{Presentation of \CCLang}

\subsection{\CCLang Origins}

% Source: "Evolving a language in and for the real world: 1991-2006", Bjarne Stroustrup

\begin{frame}{\CCLang History (1/3)}{}
  \begin{block}{Early \CCLang (1979--1998)}
    \begin{itemize}
    \item
      1979: "C with Classes", Bjarne Stroustrup, AT\&T Bell Labs
    \item
      1983: "C with Classes" $\leadsto$ \CCLang; CFront 1.0
    \item
      1985: \emph{The \CCLang Programming Language}, 1st edition, Bjarne Stroustrup
    \item
      1989: \emph{The Annotated \CCLang Reference Manual}, Bjarne Stroustrup; CFront 2.0
    \item
      1991: First ISO/IEC JTC1/SC22/WG21 meeting; CFront 3.0
    \item
      1992: \emph{Effective \CCLang}, 1st edition, Scott Meyers
    \item
      1993: Standard Template Library, Alexander Stepanov, HP Labs
    \item
      1994: \emph{The Design and Evolution of \CCLang}, Bjarne Stroustrup
    \item
      1998: \emph{Effective \CCLang}, 2nd edition, Scott Meyers
    \end{itemize}
  \end{block}
\end{frame}

\begin{frame}{\CCLang History (2/3)}{}
  \begin{block}{Standard \CCLang (1998--2011)}
    \begin{itemize}
    \item
      1998: ISO standardization, \CC{98}
    \item
      2001: \emph{Modern \CCLang Design}, Andrei Alexandrescu
    \item
      2003: \CC{03}, minor revision
    \item
      2005: \emph{Effective \CCLang}, 3rd edition, Scott Meyers
    \item
      2006: Performance Technical Report
    \item
      2007: Library Technical Report 1 (TR1)
    \end{itemize}
  \end{block}
\end{frame}

\begin{frame}{\CCLang History (3/3)}{}
  \begin{block}{Modern \CCLang (2011--)}
    \begin{itemize}
    \item
      2011: \CC{11}, major revision
      \begin{itemize}
      \item[$\to$]
        "Surprisingly, \CC{11} feels like a new language"
      \end{itemize}
    \item
      2012: Standard \CCLang Foundation
    \item
      2014: \CC{14}, minor revision; \emph{Effective Modern \CCLang}, Scott Meyers
    \item
      2017: \CC{17}, major revision
    \item
      2020: \CC{20}, next major revision of the standard
    \end{itemize}
  \end{block}
\end{frame}

\begin{frame}{Design rules (1/4)}{}
  \begin{block}{General rules}
    \begin{enumerate}
    \item
      \CCLang's evolution must be driven by real problems.
    \item
      Don't get involved in a sterile quest for perfection.
    \item
      \CCLang must be useful \emph{now}.
    \item
      Every feature must have a reasonably obvious implementation.
    \item
      Always provide a transition path.
    \item
      \CCLang is a language, not a complete system.
    \item
      Provide comprehensive support for each supported style.
    \item
      Don't try to force people to use a specific programming style.
    \end{enumerate}
  \end{block}
\end{frame}

\begin{frame}{Design rules (2/4)}{}
  \begin{block}{Design support rules}
    \begin{enumerate}
    \item
      Support sound design notions.
    \item
      Provide facilities for program organization.
    \item
      Say what you mean.
    \item
      All features must be affordable.
    \item
      It is more important to allow a useful feature than to prevent every misuse.
    \item
      Support composition of software from separately developed parts.
    \end{enumerate}
  \end{block}
\end{frame}

\begin{frame}{Design Rules (3/4)}{}
  \begin{block}{Language-technical rules}
    \begin{enumerate}
    \item
      No implicit violations of the static type system.
    \item
      Provide as good support for user-defined types as for built-in types.
    \item
      Locality is good.
    \item
      Avoid order dependencies.
    \item
      If in doubt, pick the variant of a feature that is easiest to teach.
    \item
      Syntax matters (often in perverse ways).
    \item
      Preprocessor usage should be eliminated.
    \end{enumerate}
  \end{block}
\end{frame}

\begin{frame}{Desing Rules (4/4)}{}
  \begin{block}{Low-level programming support rules}
    \begin{enumerate}
    \item
      Use traditional (dumb) linkers.
    \item
      No gratuitous incompatibilities with C.
    \item
      Leave no room for a lower-level language below C++ (except assembler).
    \item
      What you don’t use, you don’t pay for (zero-overhead rule).
    \item
      If in doubt, provide means for manual control.
    \end{enumerate}
  \end{block}
\end{frame}

\subsection{\CCLang Paradigms}

\begin{frame}{Procedural programming}{}
  \begin{definition}[Procedural programming]
  \end{definition}
  \begin{block}{Procedural programming in \CCLang}
    \CCLang is a procedural programming language.
    \begin{itemize}
    \item
    \item
    \end{itemize}
  \end{block}
\end{frame}

\begin{frame}{Object-oriented programming}{}
  \begin{definition}[Object-oriented programming]
  \end{definition}
  \begin{block}{Object-oriented programming in \CCLang}
    \CCLang is an object-oriented programming language.
    \begin{itemize}
    \item
    \item
    \end{itemize}
  \end{block}
\end{frame}

\begin{frame}{Functional programming}{}
  \begin{definition}[Functional programming]
  \end{definition}
  \begin{block}{Functional programming in \CCLang}
    \CCLang is \strong{not} a functional programming language\ldots
    \begin{itemize}
    \item
      No curryfication
    \end{itemize}
    \ldots but has elements of functional programming language.
    \begin{itemize}
    \item
      Functors and lambda functions (closures)
    \item
      \lstinline!std::function!
    \end{itemize}
  \end{block}
\end{frame}

\begin{frame}{Generic programming}{}
  \begin{definition}[Generic programming]
  \end{definition}
  \begin{block}{Generic programming in \CCLang}
    \CCLang is a generic programming language.
    \begin{itemize}
    \item
    \item
    \end{itemize}
  \end{block}
\end{frame}

\begin{frame}{Using multiple paradigms}{}
  \begin{example}
    \sourceinput{snippets/multi_paradigm.cc}

    \begin{itemize}
    \item
      Procedural: \lstinline!drawAll()!
    \item
      Object-oriented: \lstinline!shape->draw()!
    \item
      Functional: \lstinline![](const Shape *shape) \{ \}!
    \item
      Generic: \lstinline!std::vector<Shape*>! and \lstinline!std::for_each!
    \end{itemize}
  \end{example}
\end{frame}

% https://fr.wikipedia.org/wiki/Paradigme_(programmation)

\section{What you may know about \CCLang}

\subsection{Language}

%   - lvalue, rvalue, ... http://en.cppreference.com/w/cpp/language/value_category
%   - namespace
% auto
% east const, const west
% volatile
% ODR

\subsection{Standard Library}

%  new/delete -> make_unique, make_shared, vector, etc.

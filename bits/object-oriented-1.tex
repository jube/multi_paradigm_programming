\part{Object-Oriented Programming (1/2)}
\label{part:oop1}

\section{Object-Oriented Programming -- Class, Constructor, Destructor}

\subsection{Class}

% Source: https://en.cppreference.com/w/cpp/language/classes
\begin{frame}{Class}{}
  \begin{definition}[Class]
    A \strong{class} is a user-defined type whose declaration begins with \lstinline!class! or \lstinline!struct!.
  \end{definition}

  \begin{block}{Remark}
    \lstinline!struct! and \lstinline!class! are indistinguishable in C++, except that the default access mode and default inheritance mode are \lstinline!public! if class declaration uses \lstinline!struct! and \lstinline!private! if the class declaration uses \lstinline!class!.
  \end{block}
\end{frame}


% Source: https://en.cppreference.com/w/cpp/language/classes
\begin{frame}{Class members}{}
  \begin{block}{Class members}
    A class can have the following kinds of members:
    \begin{enumerate}
    \item
      Data members
      \begin{itemize}
      \item
        Non-static data members, including bit fields
      \item
        Static data members (\lstinline!static!)
      \end{itemize}
    \item
      Member functions
      \begin{itemize}
      \item
        Non-static member functions
      \item
        Static member functions (\lstinline!static!)
      \end{itemize}
    \item
      Nested types
      \begin{itemize}
      \item
        Nested classes and enumerations defined within the class definition
      \item
        Aliases of existing types, defined with \lstinline!typedef! or type alias declarations
      \end{itemize}
    \item
      Enumerators from all unscoped enumerations defined within the class
    \item
      Member templates (variable\Since{14}, class or function templates)
    \end{enumerate}
  \end{block}
\end{frame}

\begin{frame}{Class members}{}
  \begin{example}[Class members]
    \sourceinput{snippets/class_members.cc}
  \end{example}
\end{frame}


\subsection{Constructors}

\subsubsection{Constructor and member initializer list}

% Source: https://en.cppreference.com/w/cpp/language/initializer_list
\begin{frame}{Constructor}{}
  \begin{definition}[Constructor]
    A \strong{constructor} is a special non-static member function of a class that is used to initialize objects of its class type.
  \end{definition}

  \begin{block}{Remarks}
    \begin{itemize}
    \item
      Constructors have no names and cannot be called directly
    \item
      They are invoked when initialization takes place
    \end{itemize}
  \end{block}

  \begin{block}{Special kinds of constructors}
    \begin{itemize}
    \item
      Converting constructors (without \lstinline!explicit! specifier)
    \item
      Default constructor (without arguments)
    \item
      Copy constructor and move constructor (with another object of the same type as the argument)
    \end{itemize}
  \end{block}
\end{frame}

% Source: https://en.cppreference.com/w/cpp/language/initializer_list
\begin{frame}{Initialization order}{}
  \begin{block}{Initialization order}
    \begin{enumerate}
    \item
      Initialization of virtual bases
      \begin{itemize}
      \item
        Only in the most derived class of the object that's being constructed
      \item
        In depth-first left-to-right traversal of the base class declarations
      \end{itemize}
    \item
%       \label{lst:initbase}
      Initialization of direct bases
      \begin{itemize}
      \item
        In left-to-right order
      \end{itemize}
    \item
%       \label{lst:initdatamembers}
      Initialization of non-static data members
      \begin{itemize}
      \item
        In order of declaration in the class definition
      \end{itemize}
    \item
      Execution of the body of the constructor
    \end{enumerate}
%     Step~\ref{lst:initbase} to \ref{lst:initdatamembers} can be done in the \emph{member initializer list}.
  \end{block}

  % TODO: default member initializer
\end{frame}


% Source: https://en.cppreference.com/w/cpp/language/initializer_list
\begin{frame}{Member initializer list}{}
  \begin{definition}[Member initializer list]
    In the definition of a constructor of a class, a \strong{member initializer list} specifies the \emph{non-default} initializers for direct and virtual base subobjects and non-static data members.
  \end{definition}

  \begin{block}{Remark}
    The list may not be in the same order as the members but compilers issue a warning in that case.
  \end{block}
\end{frame}

\begin{frame}{Member initializer list}{}
  \begin{example}[Member initializer list]
    \sourceinput{snippets/member_initializer_list.cc}
  \end{example}
\end{frame}

\subsubsection{Converting constructor}

% Source: https://en.cppreference.com/w/cpp/language/converting_constructor
\begin{frame}{Converting constructor and explicit constructors}{}
  \begin{definition}[Converting constructor]
    A \strong{converting constructor} is not declared with the specifier \lstinline!explicit!
  \end{definition}

  \begin{block}{Converting constructors and explicit constructors}
    \begin{itemize}
    \item
      Both kind of constructors are considered in direct initialization
    \item
      Converting constructors are also considered in copy initialization (conversion)
    \end{itemize}
  \end{block}
\end{frame}

\begin{frame}{Converting constructor}{}
  \begin{example}[Converting constructor]
    \sourceinput{snippets/converting_constructor.cc}
  \end{example}
\end{frame}

\begin{frame}{Explicit constructor}{}
  \begin{example}[Explicit constructor]
    \sourceinput{snippets/explicit_constructor.cc}
  \end{example}
\end{frame}

\subsubsection{Default constructor}

% Source: https://en.cppreference.com/w/cpp/language/default_constructor
\begin{frame}{Default constructor}{}
  \begin{definition}[Default constructor]
    A \strong{default constructor} is a constructor that can be called with no arguments (either defined with an empty parameter list, or with default arguments provided for every parameter).
  \end{definition}

  \begin{block}{Default constructor}
    \begin{itemize}
    \item
      If no user-declared constructors of any kind are provided for a class, the compiler will always declare a default constructor
    \item
      A default constructor can be:
      \begin{itemize}
      \item
        Explicitely declared with \lstinline! = default! after its declaration
      \item
        Explicitely deleted with \lstinline! = delete! after its declaration
      \end{itemize}
    \end{itemize}
  \end{block}
\end{frame}

\begin{frame}{Default constructor}{}
  \begin{example}[Default constructor]
    \sourceinput{snippets/default_constructor.cc}
  \end{example}
\end{frame}


\subsubsection{Copy/move constructor/assignment}

% Source: https://en.cppreference.com/w/cpp/language/copy_constructor
\begin{frame}{Copy constructor}{}
  \begin{definition}[Copy constructor]
    A \strong{copy constructor} of class \lstinline!T! is a non-template constructor whose parameter is (generally) \lstinline!const T&!
  \end{definition}

  \begin{block}{Copy constructor}
    \begin{itemize}
    \item
      If a copy constructor is not declared, it is implicitly declared, if possible or it is implicitly deleted otherwise
    \item
      A copy constructor can be:
      \begin{itemize}
      \item
        Explicitely declared and defaulted with \lstinline! = default! after its declaration
      \item
        Explicitely deleted with \lstinline! = delete! after its declaration
      \end{itemize}
    \item
      A copy constructor is used in the following cases (\lstinline!a! and \lstinline!b! are of type \lstinline!T!):
      \begin{itemize}
      \item
        Initialization: \lstinline!T a = b;! or \lstinline!T a(b);!
      \item
        Function argument passing: \lstinline!f(a);!, where \lstinline!f! is \lstinline!void f(T t)!
      \item
        Function return: \lstinline!return a;! inside a function such as \lstinline!T f()!, where type \lstinline!T! has no move constructor
      \end{itemize}
    \end{itemize}
  \end{block}
\end{frame}

% Source: https://en.cppreference.com/w/cpp/language/move_constructor
\begin{frame}{Move constructor}{}
  \begin{definition}[Move constructor]
    A \strong{move constructor} of class \lstinline!T! is a non-template constructor whose parameter is (generally) \lstinline!T&&!
  \end{definition}

  \begin{block}{Move constructor}
    \begin{itemize}
    \item
      If a move constructor is not declared, it is implicitly declared, if possible or it is implicitly deleted otherwise
    \item
      A move constructor can be:
      \begin{itemize}
      \item
        Explicitely declared and defaulted with \lstinline! = default! after its declaration
      \item
        Explicitely deleted with \lstinline! = delete! after its declaration
      \end{itemize}
    \item
      A move constructor is used in the following cases (\lstinline!a! and \lstinline!b! are of type \lstinline!T!):
      \begin{itemize}
      \item
        Initialization: \lstinline!T a = std::move(b);! or \lstinline!T a(std::move(b));!
      \item
        Function argument passing: \lstinline!f(std::move(a));!, where \lstinline!f! is \lstinline!void f(T t)!
      \item
        Function return: \lstinline!return a;! inside a function such as \lstinline!T f()!, where type \lstinline!T! has a move constructor
      \end{itemize}
    \end{itemize}
  \end{block}
\end{frame}

\begin{frame}{Copy and move constructors}{}
  \begin{example}[Copy and move constructors]
    \sourceinput{snippets/copy_move_constructors.cc}
  \end{example}
\end{frame}

% Source: https://en.cppreference.com/w/cpp/language/copy_assignment
\begin{frame}{Copy assignment}{}
  \begin{definition}[Copy assignment]
    A \strong{copy assignment operator} of class \lstinline!T! is a non-template non-static member function with the name \lstinline!operator=! and whose parameter is (generally) \lstinline!const T&! and whose return type is \lstinline!T&!
  \end{definition}

  \begin{block}{Copy assignment}
    \begin{itemize}
    \item
      If a copy assignment is not declared, it is implicitly declared, if possible or it is implicitly deleted otherwise
    \item
      A copy assignment can be:
      \begin{itemize}
      \item
        Explicitely declared and defaulted with \lstinline! = default! after its declaration
      \item
        Explicitely deleted with \lstinline! = delete! after its declaration
      \end{itemize}
    \end{itemize}
  \end{block}
\end{frame}

% Source: https://en.cppreference.com/w/cpp/language/move_assignment
\begin{frame}{Move assignment}{}
  \begin{definition}[Move assignment]
    A \strong{move assignment operator} of class \lstinline!T! is a non-template non-static member function with the name \lstinline!operator=! and whose parameter is (generally) \lstinline!T&&! and whose return type is \lstinline!T&!
  \end{definition}

  \begin{block}{Move assignment}
    \begin{itemize}
    \item
      If a move assignment is not declared, it is implicitly declared, if possible or it is implicitly deleted otherwise
    \item
      A move assignment can be:
      \begin{itemize}
      \item
        Explicitely declared and defaulted with \lstinline! = default! after its declaration
      \item
        Explicitely deleted with \lstinline! = delete! after its declaration
      \end{itemize}
    \end{itemize}
  \end{block}
\end{frame}

\begin{frame}{Copy and move assignment}{}
  \begin{block}{Copy and move assignment}
    If both copy and move assignment operators are provided, overload resolution selects:
    \begin{itemize}
    \item
      the move assignment if the argument is an \emph{rvalue} (either a \emph{prvalue} such as a nameless temporary or an \emph{xvalue} such as the result of \lstinline!std::move!),
    \item
      the copy assignment if the argument is an \emph{lvalue} (named object or a function/operator returning lvalue reference)
    \end{itemize}
    If only the copy assignment is provided, all argument categories select it (as long as it takes its argument by value or as reference to const, since rvalues can bind to const references), which makes copy assignment the fallback for move assignment, when move is unavailable.
  \end{block}
\end{frame}

\begin{frame}{Copy and move assignments}{}
  \begin{example}[Copy and move assignments]
    \sourceinput{snippets/copy_move_assignments.cc}
  \end{example}
\end{frame}


\subsection{Destructors}

\subsubsection{Destructor}

% Source: https://en.cppreference.com/w/cpp/language/destructor
\begin{frame}{Destructor}{}
  \begin{definition}[Destructor]
    A \strong{destructor} is a special member function that is called when the lifetime of an object ends. The purpose of the destructor is to free the resources that the object may have acquired during its lifetime.

    {
      \hfill\lstinline[mathescape]!\~ $ClassName$ ()!\hfill
    }
  \end{definition}

  \begin{block}{Destruction sequence}
    \begin{enumerate}
    \item
      Execution of the body of the destructor
    \item
      Destruction of non-static data members
      \begin{itemize}
      \item
        In reverse order of declaration in the class definition
      \end{itemize}
    \item
      Destruction of direct bases
      \begin{itemize}
      \item
        In reverse order of construction, i.e. right-to-left
      \end{itemize}
    \item
      Destruction of virtual bases
      \begin{itemize}
      \item
        Only in the most derived class of the object that's being destroyed
      \end{itemize}
    \end{enumerate}
  \end{block}
\end{frame}

% Source: https://en.cppreference.com/w/cpp/language/destructor
\begin{frame}{Destructor call}{}
  \begin{block}{Destructor call}
    The destructor is called whenever an object's lifetime ends, which includes:
    \begin{itemize}
    \item
      program termination, for objects with static storage duration
    \item
      thread exit, for objects with thread-local storage duration
    \item
      end of scope, for objects with automatic storage duration and for temporaries whose life was extended by binding to a reference
    \item
      delete expression, for objects with dynamic storage duration
    \item
      end of the full expression, for nameless temporaries
    \item
      stack unwinding, for objects with automatic storage duration when an exception escapes their block, uncaught
    \item[$\to$]
      Most of the time, the destructor is called automatically!
    \end{itemize}
  \end{block}
\end{frame}

\begin{frame}{Destructor}{}
  \begin{example}[Destructor]
    \sourceinput{snippets/destructor.cc}
  \end{example}
\end{frame}


\subsubsection{Rule of Three, Rule of Five, Rule of Zero}

% Source: https://en.cppreference.com/w/cpp/language/rule_of_three
\begin{frame}{Rule of Three}{}
  \begin{definition}[Rule of Three]
    If a class requires
    \begin{itemize}
    \item
      a user-defined destructor,
    \item
      a user-defined copy constructor,
    \item
      or a user-defined copy assignment operator,
    \end{itemize}
    it almost certainly requires all three.
  \end{definition}

  \begin{block}{Reason}
    The implicitly-defined special member functions are typically incorrect if the class is managing a resource whose handle is an object of non-class type (raw pointer, POSIX file descriptor, etc), whose destructor does nothing and copy constructor/assignment operator performs a "shallow copy" (copy the value of the handle, without duplicating the underlying resource).
  \end{block}
\end{frame}

\begin{frame}{Rule of Three}{}
  \begin{example}[Rule of Three]
    \sourceinput{snippets/rule_of_three.cc}
  \end{example}
\end{frame}

% Source: https://en.cppreference.com/w/cpp/language/rule_of_three
\begin{frame}{Rule of Five}{}
  \begin{definition}[Rule of Five]
    Because the presence of a user-defined destructor, copy-constructor, or copy-assignment operator prevents implicit definition of the move constructor and the move assignment operator, any class for which move semantics are desirable, has to declare all five special member functions
  \end{definition}

  \begin{block}{Remark}
    Unlike Rule of Three, failing to provide move constructor and move assignment is usually not an error, but a missed optimization opportunity.
  \end{block}
\end{frame}

\begin{frame}{Rule of Five}{}
  \begin{example}[Rule of Five]
    \sourceinput{snippets/rule_of_five.cc}
  \end{example}
\end{frame}

% Source: https://en.cppreference.com/w/cpp/language/rule_of_three
\begin{frame}{Rule of Zero}{}
  \begin{definition}[Rule of Zero]
    Classes that have custom destructors, copy/move constructors or copy/move assignment operators should deal exclusively with ownership (which follows from the Single Responsibility Principle). Other classes should not have custom destructors, copy/move constructors or copy/move assignment operators.
  \end{definition}
\end{frame}

\begin{frame}{Rule of Zero}{}
  \begin{example}[Rule of Zero]
    \sourceinput{snippets/rule_of_zero.cc}
  \end{example}
\end{frame}

\subsubsection{Resource Acquisition Is Initialization (RAII)}

% Source: https://en.cppreference.com/w/cpp/language/raii
\begin{frame}{Resource Acquisition Is Initialization (RAII)}{}
  \begin{definition}[Resource Acquisition Is Initialization (RAII)]
    \strong{Resource Acquisition Is Initialization} (RAII), is a C++ programming technique which binds the life cycle of a resource that must be acquired before use to the lifetime of an object.
  \end{definition}

  \begin{examples}[Resources where RAII could be used]
    \begin{itemize}
    \item
       Allocated heap memory
    \item
       Thread of execution
    \item
       Open socket
    \item
       Open file
    \item
       Locked mutex
    \item
       Disk space
    \item
       Database connection
    \end{itemize}
  \end{examples}
\end{frame}

% Source: https://en.cppreference.com/w/cpp/language/raii
\begin{frame}{RAII in Practice}{}
  \begin{block}{RAII in Practice}
    \begin{itemize}
    \item
      Encapsulate each resource into a class, where
      \begin{itemize}
      \item
        the constructor acquires the resource and establishes all class invariants or throws an exception if that cannot be done
      \item
        the destructor releases the resource and never throws exceptions
      \end{itemize}
    \item
      Always use the resource via an instance of a RAII-class that either
      \begin{itemize}
      \item
        has automatic storage duration or temporary lifetime itself, or
      \item
        has lifetime bounded by the lifetime of an automatic or temporary object
      \end{itemize}
    \end{itemize}
  \end{block}

  \begin{block}{Remark}
    Classes with \lstinline!open()!/\lstinline!close()!, \lstinline!lock()!/\lstinline!unlock()!, or \lstinline!init()!/\lstinline!destroy()! member functions are typical examples of non-RAII classes
  \end{block}
\end{frame}

\begin{frame}{Resource Acquisition Is Initialization}{}
  \begin{example}[Resource Acquisition Is Initialization]
    \sourceinput{snippets/raii.cc}
  \end{example}
\end{frame}

% - Others might be more helpful in grokking what it really is, such as "Constructor Acquires, Destructor Releases" (CADRe) or "Scope-based Resource Management" (SBRM).

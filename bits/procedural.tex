\part{Procedural programming}
\label{part:proc}

\section{Procedural programming}

% definition and declaration

\subsection{Arguments}

\subsubsection{Default arguments}

\begin{frame}{Default arguments}{}
  \begin{definition}[Default arguments]
    \strong{Default arguments} are used in calls where \emph{trailing} arguments are missing.
  \end{definition}

  \begin{example}[Default arguments]
    \sourceinput{snippets/default_arguments.cc}
  \end{example}
\end{frame}

\subsubsection{Variadic arguments}

\begin{frame}{Variadic arguments}{}
  \begin{definition}[Variadic arguments]
    \strong{Variadic arguments} are used when a function accepts any number of arguments. They are indicated by a parameter of the form \lstinline!...! (ellipsis) that must appear last in the parameter list of the function declaration.
  \end{definition}

  \begin{block}{Remarks}
    \begin{itemize}
    \item
      The ellipsis can be noted \lstinline!...! (\CCLang style) or \lstinline!, ...! (C style)
      \smallskip
      \sourceinput{snippets/ellipsis.cc}
    \item
      To be able to access the arguments, the function must have at least one named paramter before the ellipsis
    \end{itemize}
  \end{block}
\end{frame}

\begin{frame}{Accessing variadic arguments}{}
  \begin{block}{Accessing variadic arguments}
    The \lstinline!<cstdarg>! header provides facilities for accessing the variadic arguments:
    \begin{itemize}
    \item
      The \lstinline!va_list! type holds the information for accessing the variadic arguments
    \item
      The \lstinline!va_start! macro enables access to the variable arguments
    \item
      The \lstinline!va_arg! macro accesses the next variadic function argument
    \item
      The \lstinline!va_end! macro ends traversal of the variadic function arguments
    \end{itemize}
  \end{block}

  \begin{block}{Default argument promotions}
    \begin{itemize}
    \item
      \lstinline!nullptr_t! is converted to \lstinline!void*!
    \item
      \lstinline!float! is converted to \lstinline!double!
    \item
      \lstinline!bool!, \lstinline!char!, \lstinline!short! are converted to \lstinline!int!
    \end{itemize}
  \end{block}
\end{frame}


\begin{frame}{Example of variadic arguments}{}
  \begin{example}
    \sourceinput{snippets/variadic_arguments.cc}
  \end{example}
\end{frame}

\subsection{Overloading}

\subsubsection{Function overloading}

\begin{frame}{Function overloading}{}
  \begin{definition}[Function overloading]
    \strong{Function overloading} is the ability to create multiple functions with the same name with different implementations. Two overloaded function must differ by the type of their parameters.
  \end{definition}

  \begin{block}{Overload resolution}
    When a function is overloaded, overload resolution is the process to select the function that will be called.
    \begin{itemize}
    \item
      Determination of the set of candidate functions after name lookup and template argument deduction
    \item
      Determination of the set of viable functions after examining arguments and parameters
    \item
      Choice of the best viable function
    \end{itemize}
    Overload resolution may fail and result in a compilation error.
  \end{block}
\end{frame}

\begin{frame}{Function overloading}{}
  \begin{example}[Function overloading]
    \sourceinput{snippets/overload.cc}
  \end{example}
\end{frame}

\subsubsection{Operator overloading}

\begin{frame}{Operator overloading}{}
  \begin{block}{Operator overloading}
    Operator overloading is a special case of function overloading. Operators that can be overloaded are:
    \begin{itemize}
    \item
      Arithmetic operators: \lstinline!+!, \lstinline!-!, \lstinline!*!, \lstinline!/!, \lstinline!\%!, \lstinline!\^!, \lstinline!\&!, \lstinline!|!, \lstinline!\~!, \lstinline!<<!, \lstinline!>>!
    \item
      Increment and decrement operators: \lstinline!++!, \lstinline!--!
    \item
      Logical operators: \lstinline+!+, \lstinline!&&!, \lstinline!||!
    \item
      Assignment operators: \lstinline!=!, \lstinline!+=!, \lstinline!-=!, \lstinline!*=!, \lstinline!/=!, \lstinline!\%=!, \lstinline!\^=!, \lstinline!\&=!, \lstinline!|=!, \lstinline!<<=!, \lstinline!>>=!
    \item
      Comparisons operators: \lstinline!==!, \lstinline+!=+, \lstinline!<!, \lstinline!>!, \lstinline!<=!, \lstinline!>=!, \lstinline!<=>!\Since{20}
    \item
      Access operators: \lstinline!->!, \lstinline!->*!, \lstinline![]!
    \item
      Special operators: \lstinline!,! (comma), \lstinline!()! (call)
    \end{itemize}
  \end{block}
\end{frame}

\begin{frame}{Overloaded operators}{}
  \begin{block}{Overloaded operators}
    \begin{center}
      \scriptsize
      \begin{tabular}{|c|c|c|l|}
        \hline
        Expression & Member function & Free function & Example \\
        \hline
        \lstinline!@a! & \lstinline!(a).operator@()! & \lstinline!operator@(a)! & \lstinline+!std::cin+ \\
                      &                             &                          & $\to$ \lstinline+std::cin.operator!()+ \\
        \hline
        \lstinline!a @ b! & \lstinline!(a).operator@(b)! & \lstinline!operator@(a,b)! & \lstinline!std::cout << 42! \\
                          &                              &                            & $\to$ \lstinline!std::cout.operator<<(42)! \\
        \hline
        \lstinline!a = b! & \lstinline!(a).operator=(b)! &                            & \lstinline!std::string s; s = "abc";! \\
                          &                              &                            & $\to$ \lstinline!std::string.operator=("abc")! \\
        \hline
        \lstinline!a(b)! & \lstinline!(a).operator()(b)! &                            & \lstinline!std::random_device r; auto n = r();! \\
                        &                               &                            & $\to$ \lstinline!r.operator()()! \\
        \hline
        \lstinline!a[b]! & \lstinline!(a).operator[](b)! &                            & \lstinline!std::map<int, int> m; m[1] = 2;! \\
                        &                               &                            & $\to$ \lstinline!m.operator[](1)! \\
        \hline
        \lstinline!a->!  & \lstinline!(a).operator->()!  &                            & \lstinline!auto p = std::make_unique<S>(); p->bar()! \\
                        &                               &                            & $\to$ \lstinline!p.operator->()! \\
        \hline
        \lstinline!a@!   & \lstinline!(a).operator@(0)!  & \lstinline!operator@(a,0)! & \lstinline!auto i = v.begin(); i++! \\
                        &                               &                            & $\to$ \lstinline!i.operator++(0)! \\
        \hline
      \end{tabular}
    \end{center}
  \end{block}
\end{frame}

\begin{frame}{Restrictions on overloaded operators}{}
  \begin{block}{Restrictions on overloaded operators}
    \begin{itemize}
    \item
      The following operators can \strong{not} be overloaded:
      \begin{itemize}
      \item
        \lstinline!::! (scope resolution)
      \item
        \lstinline!.! (member access)
      \item
        \lstinline!.*! (member access through pointer to member)
      \item
        \lstinline!?:! (ternary conditional)
      \end{itemize}
    \item
      New operators such as \lstinline!**!, \lstinline!<>!, or \lstinline!&|! can \strong{not} be created
    \item
      The overloads of operators \lstinline!&&! and \lstinline!||! lose short-circuit evaluation
    \item
      The overload of operator \lstinline!->! must either return a raw pointer, or return an object (by reference or by value) for which operator \lstinline!->! is in turn overloaded
    \item
      It is \strong{not} possible to change:
      \begin{itemize}
      \item
        the precedence of operators
      \item
        the grouping of operators
      \item
        the number of operands of operators
      \end{itemize}
    \end{itemize}
  \end{block}
\end{frame}

\begin{frame}{Operator overloading}{}
  \begin{example}[Operator overloading]
    \sourceinput{snippets/overload_operator.cc}
  \end{example}
\end{frame}

\subsection{Parameter passing}

% https://isocpp.github.io/CppCoreGuidelines/CppCoreGuidelines#Rf-conventional

\begin{frame}{Properties of types}{}
  \begin{block}{Properties of types}
    Types may be of three kinds regarding their ease of use in functions:
    \begin{enumerate}
    \item
      Easy types:
      \begin{itemize}
      \item
        Cheap to copy: \lstinline!int! and all fundamental types, \lstinline!std::tuple<bool, int*>!, \ldots
      \item
        Impossible to copy: \lstinline!std::unique_ptr<T>!, \ldots
      \end{itemize}
    \item
      Medium types:
      \begin{itemize}
      \item
        Cheap to move: \lstinline!std::string!, \lstinline!std::vector<T>!, \ldots
      \item
        Quite cheap to move: \lstinline!BigData!, \lstinline!std::array<std::vector<T>,N>!, \ldots
      \item
        Unknown cost: \lstinline!unknown::Type!, \ldots
      \end{itemize}
    \item
      Hard types:
      \begin{itemize}
      \item
        Expensive to move: \lstinline!std::array<BigData,N>!, \ldots
      \end{itemize}
    \end{enumerate}
  \end{block}
\end{frame}

\begin{frame}{Output only values}{}
  \begin{block}{Output only values}
    For output only values, prefer:
    \begin{itemize}
    \item
      Return values for easy and medium types \\
      \lstinline!X f();!
%       \begin{itemize}
%       \item
%         In case of multiple return values, use a \lstinline!struct! or \lstinline!std::tuple! \\
%         \lstinline!std::tuple<X,Y> f();!
%       \end{itemize}
    \item
      Output parameter (reference to non-const) for hard types \\
      \lstinline!void f(X& out);!
    \end{itemize}
  \end{block}

  \begin{example}[Output only values]
    \sourceinput{snippets/parameter_passing_out.cc}
  \end{example}
\end{frame}

\begin{frame}{Input only values}{}
  \begin{block}{Input only values}
    For input only values, prefer:
    \begin{itemize}
    \item
      Passing by value for easy types \\
      \lstinline!void f(T in);!
    \item
      Passing by const reference for medium and hard types
      \lstinline!void f(const T& in);!
    \end{itemize}
  \end{block}

  \begin{example}[Input only values]
    \sourceinput{snippets/parameter_passing_in.cc}
  \end{example}
\end{frame}

\begin{frame}{Input/Output values}{}
  \begin{block}{Input/Output values}
    For input/output values, prefer:
    \begin{itemize}
    \item
      Passing by reference to non-const for any type \\
      \lstinline!void f(T& inout);!
    \end{itemize}
  \end{block}

  \begin{example}[Input/Output values]
    \sourceinput{snippets/parameter_passing_inout.cc}
  \end{example}
\end{frame}

% parameter passing and reference


\subsection{Argument-dependent lookup}

% http://en.cppreference.com/w/cpp/language/adl

\subsection{Constant expression functions}

% constexpr

% UDL
% string ids

% mangling

% copy elision


% \begin{frame}{}{}
%   \begin{block}{}
%     \begin{itemize}
%     \item
%     \item
%     \end{itemize}
%   \end{block}
% \end{frame}
